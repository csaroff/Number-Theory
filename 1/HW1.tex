%
% Elizabeth Wilcox
% MAT 373: Number Theory
%
\documentclass[11 pt]{article}

%  packages
\usepackage{amssymb, amsmath, amsthm, enumerate, graphicx, multirow, float, color}
\usepackage{tikz}
\usepackage{pifont}
\usetikzlibrary{arrows}
%  margins
\textwidth = 6.5 in 
\textheight = 9 in 
\oddsidemargin = 0.0 in 
\evensidemargin = 0.0 in 
\topmargin = 0.0 in 
\headheight = 0.0 in 
\headsep = 0.0 in 

%  theorems, corollaries, etc.
\theoremstyle{definition}
\newtheorem{thm}{Theorem}
\newtheorem{cor}[thm]{Corollary}
\newtheorem{lem}[thm]{Lemma}
\newtheorem{prop}[thm]{Proposition}

\theoremstyle{definition}
\newtheorem*{example}{Example}
\newtheorem{dfn}[thm]{Definition}

\renewcommand{\qedsymbol}{\vrule height 5pt width 4pt depth -1pt}


% spacing shortcuts
\def\skip{\vspace{10 pt}}
\def\medskip{\vspace{15 pt}}

% text shortcuts
\def\ie{{i.e.,}\ }
\def\eg{{e.g.,}\ }
\def\bbar#1{\overline{#1}}

%  set shortcuts
\def\QQ{\mathbb{Q}}
\def\ZZ{\mathbb{Z}}
\def\RR{\mathbb{R}}
\def\CC{\mathbb{C}}
\def\NN{\mathbb{N}}

% number theory shortcuts
\def\divides{\big |}

% logic shortcuts
\def\iff{\Leftrightarrow}
\thispagestyle{empty}
\begin{document}

\centerline{\bf Math 373  \hfill HW \#1}
\centerline{\bf Name: Chaskin Saroff \hfill Due: Wed. 2/4}

\begin{enumerate}
\item The {\it Fibonacci numbers $f_n$} are given by the recurrence
%
$$
f_0 = 0, f_1 = 1, f_{n+1} = f_{n-1} + f_n \textrm{ for all } n \geq 2.
$$
%
The first few Fibonacci numbers are: $0, 1, 1, 2, 3, 5, 8, 13, \dots$  
	\begin{enumerate}
	\item Prove that $\sum\limits_{i = 1}^n f_i = f_{n+2} - 1$ for all $n \geq 1$.
        
        \begin{proof}

            \

            Base Case:\\
            Let $n=1$. \\
            \begin{align} 
                \sum\limits_{i = 1}^1 f_i &= f_{1} = 1 \\
                                            &= 2-1 \\
                                            &=f_{n+2} - 1 \checkmark
            \end{align}
            Inductive Step: \\
            Let P($n$) be the statement $\sum\limits_{i = 1}^n f_i = f_{n+2} - 1$\\
            Assume P($n$) (We will show P($n+1$)).
            \begin{align}
                         &\sum\limits_{i = 1}^n f_i = f_{n+2} - 1 \\
                \implies &\sum\limits_{i = 1}^n f_i + f_{n+3} = f_{n+2} + f_{n+3} - 1 \\
                \implies &\sum\limits_{i = 1}^n f_i + (f_{n+1} + f_{n+2}) - f_{n+2} = f_{n+3} - 1 \\
                \implies &\sum\limits_{i = 1}^{n+1} f_i = f_{n+3} - 1 \\
                \implies & P(n+1) \text{ is true.} 
            \end{align}
            Therefore $\sum\limits_{i = 1}^n f_i = f_{n+2} - 1$ for all $n \in \ZZ^+$.
        \end{proof}
        \pagebreak%---------------------------------------------------------------------------------

	\item Prove that $\sum\limits_{i=1}^n f_i^2 = f_nf_{n+1}$ for all positive integers $n$.
        
        \begin{proof}

            \

            Base Case:\\
            Let $n=1$. \\
            \begin{align} \setcounter{equation}{0}
                \sum\limits_{i=1}^1 f_{i}^2 &= f_i^2\\
                                          &= 1^2\\
                                          &= 1 \times 1\\
                                          &= f_1 f_2 \checkmark
            \end{align}%End Base Case

            Inductive Step: \\
            Let P($n$) be the statement $\sum\limits_{i=1}^n f_i^2 = f_n f_{n+1}$\\
            Assume P($n$) (We will show P($n+1$)).
            \begin{align}
                &\sum\limits_{i=1}^n f_{i}^2 = f_nf_{n+1}\\
                \implies &\sum\limits_{i=1}^n f_i^2 + f_{n+1}^2 = f_nf_{n+1}+f_{n+1}^2\\
                \implies &\sum\limits_{i=1}^{n+1} f_i^2 = f_{n+1}(f_n+f_{n+1})\\
                \implies &\sum\limits_{i=1}^{n+1} f_i^2 = f_{n+1}f_{n+2}\\
                \implies & P(n+1) \text{ is true.}
            \end{align}
            Therefore $\sum\limits_{i=1}^n f_i^2 = f_nf_{n+1}$ for all $n \in \ZZ^+$
        \end{proof}
	\end{enumerate}
\pagebreak%-----------------------------------------------------------------------------------
\item Prove that $2^{2n-1} + 1$ is divisible by 3 for every positive integer $n$.\\
        ie. Prove that $2^{2n-1} + 1 = 3x$ for some $x \in \ZZ$ 
        \begin{proof}

            \

            Base Case:\\
            Let $n=1$. \\
            \begin{align} \setcounter{equation}{0}
                2^{2n-1}+1 &= 2^{2(1)-1}+1\\
                           &= 2+1\\
                           &= 3(1) \checkmark
            \end{align}%End Base Case

            Inductive Step: \\
            Let P($n$) be the statement $2^{2n-1} + 1 = 3x$ for some $x \in \ZZ$\\
            Assume P($n$) (We will show P($n+1$)).
            \begin{align}
                &2^{2n-1} + 1 = 3x\\
                \implies &2^2(2^{2n-1}) + 1 = 2^2(3x)\\
                \implies &2^{2n-1+2}+2^2 = 2^2(3x)\\
                \implies &2^{2n+2-1}+2^2 -3 = 2^2(3x) -3\\
                \implies &2^{2(n+1)-1}+1 = 4(3)x-3\\
                \implies &2^{2(n+1)-1}+1 = 3(4x-1)\\
                         & \text{Let $y=4x-1$.}\\
                \implies & \text{$y \in \ZZ$ (by the closure of $\ZZ$ under $+,*$.)}\\
                \implies &2^{2(n+1)-1}+1 = 3(y)\\
                \implies & P(n+1) \text{ is true.}
            \end{align}%End Inductive Step
            Therefore $2^{2n-1} + 1$ is divisible be 3 for every $n \in \ZZ$ by definition of divides
        \end{proof}
        \pagebreak%---------------------------------------------------------------------------

\item Let $a, b, c \in \ZZ$.  Prove: If $a \divides b$ and $a + b = c$ then $a \divides c$.
    \begin{proof}
        \begin{align} \setcounter{equation}{0}
            &\text{Let $a,b,c \in \ZZ$.}\\\notag
            &\text{Let $a|b$ be true}\\\notag
            &\text{Let $a+b=c$ }\\\notag
            \implies &b=ax \text{ for some $x \in \ZZ$}\\
            \implies &a+ax=c\\
            \implies &a(1+x)=c\\
                     &\text{let $y=1+x$}\\
            \implies &\text{$y \in \ZZ$ by closure of $\ZZ$ under $+$}\\
            \implies &a(y) = c
        \end{align}
        Therefore $a|c$ by definition of divides.
    \end{proof}
    \pagebreak%-------------------------------------------------------------------------------
\item Let $n \in \ZZ$ be such that $5 \divides (n + 2)$.  Which of the following expressions are divisible by 5?
%
$$
n^2 - 4, n^2 + 8n + 7, n^4 - 1, n^2 - 2n
$$
%   
    Let's consider what it means for $5$ to divide $(n+2)$
    \begin{align*}
        &n+2=5x \text{ for some $x \in \ZZ$ (by definition of divides.)}\\
        \implies  &n=5x-2
    \end{align*}

    \begin{align} \setcounter{equation}{0}
        n^2 -4 &= (5x-2)^2-4\\
               &= 25x^2-10x+4-4\\
               &= 5(5x-2)\checkmark
    \end{align}
    Five divides $n^2-4$ because $n^2-4$ is equal to 5 times an integer.
    \begin{align} \setcounter{equation}{0}
        n^2+8n+7 &= (5x-2)^2+8(5x-2)+7\\
                 &= 25x^2-10x+4+40x-16+7\\
                 &= 25x^2+30x-5\\
                 &= 5(5x^2+6x-1)\checkmark
    \end{align}
    $n^2+8n-7$ is also divisible by 5.
    \begin{align} \setcounter{equation}{0}
        n^4-1 &= [n^2+1][n^2-1]\\
              &= [(5x-2)^2+1][(5x-2)^2-1]\\
              &= [25x^2-10x+5][(5x-2)^2-1]\\
              &= 5[5x^2-2x+1][(5x-2)^2-1]\checkmark
    \end{align}
    $n^4-1$ is divisible by 5.
    \begin{align} \setcounter{equation}{0}
               &\text{let $n=3$}\\
               &\text{then $5\divides(3+2)$}\\
        n^2-2n &= 3^2-2(3)\\
               &= 9-6\\
               &=3 \text{ \ding{55}}
    \end{align}
    but $n^2-2n$ is not divisible by $5$ for $n=1$

    \pagebreak%-------------------------------------------------------------------------------


\item Let $n \in \NN$. Prove or disprove, and salvage: A product of $n$ or more \emph{unique consecutive} integers is divisible by $n$.
    
\item {\bf (Sage.)} Assume that $a, b \in \NN$.
	\begin{enumerate}
	\item Write a program to determine if $a$ divides $b$.  You cannot use the built-in functions for modular arithmetic or division; these built-in function relies on a program like what you're trying to write!
	\item Write a program that lists positive divisors of $b$ and determines if $b$ is perfect, deficient, or abundant.  You may use either the program in (a) or Sage's built-in commands \textcolor{blue}{x.divides(y)} or \textcolor{blue}{mod(x,y)} -- but not the Sage command for listing divisors or for the sum of divisors.
	\end{enumerate}


\end{enumerate}
\end{document}
