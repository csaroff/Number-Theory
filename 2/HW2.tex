%
% Elizabeth Wilcox
% MAT 373: Number Theory
%
\documentclass[11 pt]{article}

%  packages
\usepackage{amssymb, amsmath, amsthm, enumerate, graphicx, multirow, float, color}
\usepackage{tikz, enumitem}
\usetikzlibrary{arrows}

%  margins
\textwidth = 6.5 in 
\textheight = 9 in 
\oddsidemargin = 0.0 in 
\evensidemargin = 0.0 in 
\topmargin = 0.0 in 
\headheight = 0.0 in 
\headsep = 0.0 in 

%  theorems, corollaries, etc.
\theoremstyle{definition}
\newtheorem{thm}{Theorem}
\newtheorem{cor}[thm]{Corollary}
\newtheorem{lem}[thm]{Lemma}
\newtheorem{prop}[thm]{Proposition}

\theoremstyle{definition}
\newtheorem*{example}{Example}
\newtheorem{dfn}[thm]{Definition}

\renewcommand{\qedsymbol}{\vrule height 5pt width 4pt depth -1pt}
%\renewcommand{\theenumi}{\Alph{enumi}}                       


% spacing shortcuts
\def\skip{\vspace{10 pt}}
\def\medskip{\vspace{15 pt}}

% text shortcuts
\def\ie{{i.e.,}\ }
\def\eg{{e.g.,}\ }
\def\bbar#1{\overline{#1}}

%  set shortcuts
\def\QQ{\mathbb{Q}}
\def\ZZ{\mathbb{Z}}
\def\RR{\mathbb{R}}
\def\CC{\mathbb{C}}
\def\NN{\mathbb{N}}

% number theory shortcuts
\def\divides{\big |}
\def\gcd#1#2{\textrm{gcd}({#1},{#2})}
\def\ord#1{|{#1}|}

% logic shortcuts
\def\iff{\Leftrightarrow}
\thispagestyle{empty}
\begin{document}

\centerline{\bf Math 373  \hfill HW \#2}
\centerline{\bf NAME: Chaskin Saroff\hfill DUE: Wed. 2/11}


\

\noindent
{\bf Latex Exercises}: Type up your solutions to \#3 using latex.

\begin{enumerate}
\item For each of the following, use the Euclidean Algorithm to find $\gcd a b$ and determine $x, y \in \ZZ$ such that $\gcd a b = ax + by$:
%
$$
a = 7700, b = 2233; a = 455, b = 1235; 
$$
%
Show your work -- not just the answer.

\item In each of the following, apply the Division Algorithm to find $q$ and $r$ such that $a = bq + r$ and $0 \leq r \leq \ord b$:
%
$$
a = 300, b = 17; a = 729, b = 31; a = 300, b = -17; a = 389, b = 4.
$$

\pagebreak%------------------------------------------------------------------------------------------------------------------------
\item Let $a, b, c \in \ZZ$.  
	\begin{enumerate}
	\item 
        \begin{proof} If $a \divides b$ and $c \divides d$, then $ac \divides bd$\\
            Let $a,b,c,d \in \ZZ$.\\
            Let $a \divides b$\\
            Let $c \divides d$\\
            $\implies b = ax$ for some $x \in \ZZ$ by definition of divides.\\
            $\implies d = cy$ for some $y \in \ZZ$ by definition of divides.\\
            $\implies bd = axcy$ for some $x,y \in \ZZ$ by definition of divides.\\
            $\implies bd = (ac)xy$ for some $x,y \in \ZZ$ by definition of divides.\\
            Let $z=xy$\\
            Then $z \in \ZZ$ by closure of $\ZZ$ under $*$\\
            $\implies bd=ac(z)$\\
            Therefore $ac \divides bd$ by definition of divides.
        \end{proof}
        \pagebreak%------------------------------------------------------------------------------------------------------------
        \item 
        \begin{proof} $b \divides a$ if and only if $\gcd a b = \ord b$.\\
                Recall the gcd condition.\\
                $gcd(a,b)=d$ if and only if all of the following are true:\\

                \begin{enumerate}[label=(\alph*)]
                    \item $d>0$
                    \item $d \divides a$
                    \item $d \divides b$
                    \item $m \divides d$ for all common divisors, m, of a and b.
                \end{enumerate}
            \begin{enumerate}
                \item We will first show that $b \divides a \implies gcd(a,b)=\ord b$\\
                    Let $a,b \in \ZZ$\\
                    Let $b|a$\\
                    $b \neq 0$ because $0 \divides n$ is false $\forall n \in \ZZ$\\
                    (a) Since $b \in \ZZ$ and $b \neq 0$, $\ord b > 0 \checkmark$\\
                    \\
                    $b \divides a \implies a=bc$ for some $c \in \ZZ$ by definition of divides.\\
                    Then $a = (-b)(-c)$.\\
                    (b) $\ord b \divides a \checkmark$\\
                    \\
                    To show $\ord b \divides b$:\\
                    $b = b(1)$ and $b = (-b)(-1)$\\
                    (c) $\ord b \divides b$ by definition of divides. $\checkmark$\\
                    \\
                    (d) Since $\ord b$ is the greatest divisor of b and $\ord b$ is a common divisor of a and b,\\
                        then $\ord b = gcd(a,b) \checkmark$\\
                    \item Let $gcd(a,b) = \ord b$.
                        Then $\ord b \divides a \implies b \neq 0$\\
                        $a = \ord b (c)$ for some $c \in \ZZ$\\
                        Case 1. $b > 0$\\
                            $a = bc$.\\
                        Case 2. $b < 0$.\\
                            $a = b(-c)$.\\
                        in either case, $b \divides a$ by definition.\\
                        Then $\ord b \divides a$ by definition.
            \end {enumerate}
            Therefore $b \divides a$ if and only if $gcd(a,b)= \ord b$
        \end{proof}
    \pagebreak%--------------------------------------------------------------------------------------------------------------------
	\item Show that if $c > 0$ then $\gcd {ac}{bc} = c (\gcd a b)$.
        \begin{proof}
        \end{proof}
	\end{enumerate}
    \pagebreak%--------------------------------------------------------------------------------------------------------------------

\item {\bf (Sage.)} Assume that $m, n \in \NN$.  Recall the Fibonacci numbers $f_n$ from the first homework.
	\begin{enumerate}
	\item Using Sage's built-in command for the greatest common divisor, write a program that will take an input $c \geq 1$ and then compute $\gcd {f_n}{f_m}$ for all $n, m$ such that $1 \leq n < m \leq c$.  It's ok to use the command \textcolor{blue}{fibonacci(n)}, which computes $f_n$.
	\item Print out the greatest common divisors between all pairs of Fibonacci numbers $f_n$, $f_m$ with $1 \leq n < m \leq 50$.  Do you notice any patterns?  Make two reasonable, meaningful conjectures regarding the greatest common divisors.  For example, can you make a conjecture about $\gcd {f_n}{f_{n+1}}$?  
	\item[]
	\item[] Can you prove your conjectures? Try!
	\end{enumerate}
	
\end{enumerate}
\end{document}
